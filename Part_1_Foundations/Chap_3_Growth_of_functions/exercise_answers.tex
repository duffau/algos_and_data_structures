\documentclass{article}
\usepackage{amsmath}
\usepackage{amssymb}


\title{Exercise answers - Chapter 3 - Growth of functions}
\author{Christian Duffau-Rasmussen}

\begin{document}

\maketitle

\subsection*{Exercises}

\subsubsection*{Exercise 3.1-1}

Defining $h(n) = \max(f(n), g(n))$. Since $f$ and $g$ are assymptotically, non-negative it holds that, $f(n)+ g(n) \leq 2h(n)$ and $f(n) + g(n)\geq h(n)$ for all $n>n_0$, for some $n_0$. So, we have,
\[
\frac{1}{2}(f(n) + g(n)) \leq h(n) \leq f(n) + g(n)\quad\forall n>n_0
\]
which ensures that $h(n)$ is a member of $\Theta(f(n) + g(n))$, with constants $c_1=1/2$ and $c_2=1$. 

\subsubsection*{Exercise 3.1-2}

For $f(n)=(n+a)^b$ to be a member of the set $\Theta(n^b)$, there must exist constants $c_1$ and $c_2$ such that,
\begin{equation} \label{eq:power}
c_1n^b \leq (n+a)^b \leq c_2n^b 
\end{equation} 
for all $n$ greater than some lower bound $n_0$.

First we notice that for some value of $n$, $n+a$ will become positive, and $f$ will we monotonically growing in $n$, from that point on.
We also notice that $(n+a)^b\rightarrow n^b$ as $n\rightarrow\infty$. So \eqref{eq:power} will hold for some $n_0$, with $c_1=1-\epsilon$ and $c_2=1+\epsilon$ for some positive $\epsilon$, hence $(n+a)^b=\Theta(n^b)$.

\subsubsection*{Exercise 3.1-5}

Theorem 3.1 states, for any two functions $f(n)$ and $g(n)$, we have $f(n)= \Theta(g(n))$ if and only if $f(n) = O(g(n))$ and $f(n) = \Omega(g(n))$.

The theorem follows from the definitions of $O(\cdot)$, $\Omega(\cdot)$ and $\Theta(\cdot)$. 

We have that $f(n)$ is $\Theta(g(n))$ if only if it's bounded from below by $c_1g(n)$ and bounded from above by $c_2g(n)$ for all $n$ above a given threshold $n_0$. If $f(n)$ is $O(g(n))$ it is bounded from below by $c_1g(n)$ and if $f(n)$ is $\Omega(g(n))$ it's bounded from above by $c_2g(n)$, for all $n$ above some threshold, hence if $f$ is both $O(g)$ and $\Omega(g)$ it must be $\Theta(g)$.

\subsubsection*{Exercise 3.2-7}

The Finbonacci sequence is given by,
\begin{align*}
F_0 &= 0\\
F_1 &= 1\\
F_i &= F_{i-1} + F_{i-2}\qquad i\geq 2.
\end{align*}
the Golden ratio and it's conjugate are,
\[
\phi = \frac{1+\sqrt{5}}{2}\qquad\hat{\phi} = \frac{1-\sqrt{5}}{2}.
\]

The claims is that the $i$-th Fibonnaci is given by the function,
\[
f(i) = \frac{\phi^i-\hat{\phi}^i}{\sqrt{5}},
\]
that is, $F_i = f(i)$ for all $i\geq0$.
\paragraph{Basis step}
We see that,
\begin{align*}
f(0) &= \frac{\phi^0-\hat{\phi}^0}{\sqrt{5}} = 0 = F_0\\
f(1) &= \frac{\phi^1-\hat{\phi}^1}{\sqrt{5}} = \frac{(1+\sqrt{5}-1+\sqrt{5})/2}{\sqrt{5}} = 1 = F_1.\\
\end{align*}


\paragraph{Induction step} Assuming that $F_{i-1}=f(i-1)$ and $F_{i-2}=f(i-2)$ we have,
\[
F_i = f(i-1)+ f(i-2)
\]
we would like to show that $f(i)=f(i-1)+ f(i-2)$ for all $i\geq2$, then by induction the claims $f(i) = F_i$ will be true.

The golden ratio and it's conjugate are solutions to $x^2 = x+1$ so we have,
\begin{align*}
\phi^{i-2} &= \frac{\phi^i}{\phi+1} \\
\hat{\phi}^{i-2} &= \frac{\hat{\phi}^i}{\hat{\phi}+1} \\
\end{align*}


\end{document}
