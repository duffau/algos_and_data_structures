\documentclass{article}
\usepackage{amsmath}
\usepackage{amssymb}


\title{Exercise answers - Chapter 3 - Growth of functions}
\author{Christian Duffau-Rasmussen}

\begin{document}

\maketitle

\subsubsection*{Exercise 3.1-1}

Defining $h(n) = \max(f(n), g(n))$. Since $f$ and $g$ are assymptotically, non-negative it holds that, $f(n)+ g(n) \leq 2h(n)$ and $f(n) + g(n)\geq h(n)$ for all $n>n_0$, for some $n_0$. So, we have,
\[
\frac{1}{2}(f(n) + g(n)) \leq h(n) \leq f(n) + g(n)\quad\forall n>n_0
\]
which ensures that $h(n)$ is a member of $\Theta(f(n) + g(n))$, with constants $c_1=1/2$ and $c_2=1$. 

\subsubsection*{Exercise 3.1-2}

For $f(n)=(n+a)^b$ to be a member of the set $\Theta(n^b)$, there must exist constants $c_1$ and $c_2$ such that,
\begin{equation} \label{eq:power}
c_1n^b \leq (n+a)^b \leq c_2n^b 
\end{equation} 
for all $n$ greater than some lower bound $n_0$.

First we notice that for some value of $n$, $n+a$ will become positive, and $f$ will we monotonically growing in $n$, from that point on.
We also notice that $(n+a)^b\rightarrow n^b$ as $n\rightarrow\infty$. So \eqref{eq:power} will hold for some $n_0$, with $c_1=1-\epsilon$ and $c_2=1+\epsilon$ for some positive $\epsilon$, hence $(n+a)^b=\Theta(n^b)$.

\subsubsection*{Exercise 3.1-5}

Theorem 3.1 states, for any two functions $f(n)$ and $g(n)$, we have $f(n)= \Theta(g(n))$ if and only if $f(n) = O(g(n))$ and $f(n) = \Omega(g(n))$.

The theorem follows from the definitions of $O(\cdot)$, $\Omega(\cdot)$ and $\Theta(\cdot)$. 

We have that $f(n)$ is $\Theta(g(n))$ if only if it's bounded from below by $c_1g(n)$ and bounded from above by $c_2g(n)$ for all $n$ above a given threshold $n_0$. If $f(n)$ is $O(g(n))$ it is bounded from below by $c_1g(n)$ and if $f(n)$ is $\Omega(g(n))$ it's bounded from above by $c_2g(n)$, for all $n$ above some threshold, hence if $f$ is both $O(g)$ and $\Omega(g)$ it must be $\Theta(g)$.

\subsubsection*{Exercise 3.2-1}


\end{document}
