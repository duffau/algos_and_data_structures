\documentclass{article}
\usepackage[noend]{algpseudocode}
\usepackage{amsmath}
\usepackage{amssymb}

\algnewcommand\algorithmicto{\textbf{to}}
\algrenewtext{For}[3]{\algorithmicfor\ $#1 \gets #2$ \algorithmicto\ $#3$ \algorithmicdo}
\let\oldReturn\Return
\renewcommand{\Return}{\State\oldReturn}

\providecommand{\ceil}[1]{\left \lceil #1 \right \rceil }
\providecommand{\floor}[1]{\left \lfloor #1 \right \rfloor }

\title{Exercise answers - Chapter 2 - Getting Started}
\author{Christian Duffau-Rasmussen}

\begin{document}

\maketitle

\subsubsection*{Exercise 2.1-3 Linear Search}

\begin{algorithmic}[1]
\Procedure{Linear-Search}{A, v}
\For{i}{1}{\text{length}[A]}
	\If{$A[i]==v$}
		\Return $i$
	\EndIf
\EndFor
\Return NILL
\EndProcedure
\end{algorithmic}

\paragraph{Proof of correctness}

To prove linear search is correct, I formulate a \emph{loop invariant} that need to hold true at \emph{initialization}, during \emph{maintainance} and at \emph{termination}.

\subparagraph{Loop invariant} At the start of loop iteration $i$, the value $v$ is not in the sub-array $A[1..i-1]$. In formal terms, $v \notin \{ A[j] \vert j\in\{1,...,i-1\} \}$

\subparagraph{Initialization} At initialization $i=0$, so the sub-array $A[1..i-1]$ is empty. So $v$ is trivially not a member of that array, and the loop invariant holds.

\subparagraph{Maintenance} Assuming that $v$ is not in the sub-array $A[1..i-1]$ at the start of iteration $i$, then two outcomes are possible during the iteration. Either, $A[i] == v$ and the loop terminates, or $A[i]\neq v$ and $v$ is not an element of the array $A[1..i]$ which is the loop invariance condition for iteration $i+1$.

\subparagraph{Termination} The algoritm terminates in two different ways. The first happens when $A[i] == v$, assuming $v \notin A[1..i-1]$ before termination, then the loop invariant still holds. Again, assuming $v \notin A[1..i-1]$ before termination. In the second case we have that $i=n+1$, in that case $v \notin A[1..i-1]=A[1..n]$, which is the entire array, and the procedure returns NILL.

\subsubsection*{2.2-3 Running time of Linear search}
Assuming the index of correct element is uniformlly distributed from 1 to $n$, the expected index is $E[i] = n/2$. So the average case running time must be $c_1\frac{n}{2} + c_2$, where $c_1$ is the running time of each comparison $A[i]==v$ and the loop increment,  and $c_2$ is the running time of the return statement. In the worst case the algorithm needs to run through all $n$ elements as well as returning NILL, so the running time is $c_1n + c_2$.  
Both the average and worst case running times of linear search are $\Theta(n)$.

\subsubsection*{Exercise 2.3-3 Linear Reccurence}

To show that the reccursion 
\begin{equation*}
T(n) = \begin{cases}
2 &           \text{if $n=2$}\\
2T(n/2) + n & \text{if $n=2^k$, for $k>1$}
\end{cases} 
\end{equation*}
is $T(n)=n \lg n$. Assuming that $n$ is a power of two we have $n=2^k$ for $k\in{1,2,\ldots}$. For $k=1$ we have that $T(n)=T(2)=2$ by definition of $T$.  
\[
2 \lg(2) = 2 = T(2).
\]
Hence the claim holds for $k=1$.

Next, assuming that $T(2^{k-1})= 2^{k-1}\lg 2^{k-1} = (k-1)2^{k-1}$ we have that,
\begin{equation*}
\begin{split}
T(n)   & = 2T(n/2) + n \\
       & = 2T(2^k/2) + 2^k \\
       & = 2T(2^{k-1}) + 2^k \\
       & = 2((k-1)2^{k-1}) + 2^k \\
       & = k2^k \\
       & = 2^k \lg(2^k) = n \lg n
\end{split}
\end{equation*}
so the claim also holds for $k$ given it holds for $k-1$, at since it holds true for $k=1$ it holds for any $k\geq1$.\hfill $\square$

\subsubsection*{Exercise 2.3-5 Binary search}

\begin{algorithmic}[1]
\Procedure{Binary-Search}{A, p, r, v}
\If{$p\leq r$}
	\State $m=p + \floor{\frac{r - p}{2}}$
	\If{$A[m]==v$}
		\Return m
	\ElsIf{$A[m] > v$}
		\Return \Call{Binary-Search}{A, p, m - 1, v}
	\ElsIf{$A[m] < v$}
		\Return \Call{Binary-Search}{A, m +1, r, v}
	\EndIf
\EndIf
\Return NILL
\EndProcedure
\end{algorithmic}

In the worst case, binary search needs to split the array in two, untill the last split goes from two elements to a single one. A way of assessing the running time is to understand how many possible splits can be made. Assuming the length of the array is $n=2^k$, the length after the first split is simply $n/2$. Finding the length of the consequent halfed arrays, amounts to raising the \emph{half} to the nummber of splits and multiplying by the initial size, $n$. So, to find the number of splits which result in the length of 1, we need to solve the following,
\[
1 = \frac{n}{2^x} \quad\Leftrightarrow\quad 2^x = n \quad\Leftrightarrow\quad x = \lg n. 
\]
So we can conclude, that the recursion calls itself at most  $\lg n$ times, this is an indication of the worst case running time being $\Theta(\lg n)$.

\subsubsection*{Problem 2-3 Horner's rule}

\paragraph{a. Running time}
The running time of each line in Horner's algoritm is given by,

\begin{algorithmic}[1]
\State $y \gets 0$ \Comment{Running time: $c_1$}
\State $i \gets n$ \Comment{Running time: $c_1$}
\While{$i\geq0$}   \Comment{Running time: $c_2$ repeated $n+2$ times.}
	\State $y\gets a_i + x \cdot y$  \Comment{Running time: $c_1 + c_{+} + c_{\bullet}$ repeated $n+1$ times.}
	\State $i \gets i - 1$ \Comment{Running time: $c_1 + c_{-}$ repeated $n+1$ times.}
\EndWhile
\end{algorithmic}

The total running time is $T(n) = 2c_1 + c_2(n+2)+(2c_1 + c_{+} + c_{\bullet} + c_{-})(n+1)$, that is
\[
T(n) = C_1 + C_2n
\]
with $C_1 = 4c_1 + 2c_2 + c_{+} + c_{\bullet} + c_{-}$ and $C_2=2c_1 + c_2 + c_{+} + c_{\bullet} + c_{-}$. So, it's clear that $T(n)$ is a linear function of $n$ so $T(n) = \Theta(n)$.

\paragraph{b. Naive polynomial evaluation}

\begin{algorithmic}[1]
\State $y \gets 0$ 
\State $z \gets 1$ 
\State $i \gets 0$ 
\State $j \gets 0$ 
\While{$i\leq n$} 
	\While{$j < i$} \Comment{This inner loop is evaluated $\sum_{k=1}^{n+1} k$ times.}
		\State $z \gets z\cdot x$
		\State $j \gets j + 1$
	\EndWhile
	\State $y\gets y + a_i \cdot z$
	\State $i \gets i + 1$
	\State $z \gets 1$ 
	\State $j \gets 0$ 

\EndWhile
\end{algorithmic}

The inner loop is called $\sum_{k=1}^{n+1} k= (n+1)(n+2)/2 = n^2+3n+2$ times, and the evaluation in the outer loop are called $n$ times, so the running time is no more than,  
\[
T(n) = C_0 + C_1n + C_2n^2
\]
for the appropriate constants. So for the naive implementation $T(n)=\Theta(n^2)$.

\paragraph{c. Proof of correctness}
Consdering the loop invariant,
\[
y = \sum_{k=0}^{n-(i+1)}a_{k+i+1}x^k.
\]
\subparagraph{Initialization} We have $i=n$ so the above expression is $y = \sum_{k=0}^{-1} a_{k+n+1}x^k = 0$, which is in line with the assigned value to $y$ of 0 in line 1 of the algorithm. So the loop invariant is true at initialization.

\subparagraph{Maintainance} During maintenance assuming $i = j+1$ with $0\leq j+1<n$ and sssuming $y$ is given by
\[
y = \sum_{k=0}^{n-(j+2)}a_{k+j+2}x^k = a_{j+2} + a_{j+3}x + a_{j+4}x^2 + \ldots + a_nx^{n-(j+2)}
\]
at the start of the loop, then at the start of the next iteration, $i=j$, $y$ is given by
\begin{equation*}
\begin{split}
y \gets & a_{j+1} + x\cdot y \\
      = & a_{j+1} + x\left(a_{i+2} + a_{j+3}x + a_{j+4}x^2 + \ldots + a_nx^{n-(j+2)} \right)\\
      = & a_{j+1} + a_{i+2}x + a_{j+3}x^2 + a_{j+4}x^3 + \ldots + a_nx^{n-(j+1)}\\
\end{split}
\end{equation*}
which corresponds to the loop invariant at iteration $i=j$. So, the loop invariant holds in the maintainance step as well.

\subparagraph{Termination}
At termination $i=-1$ so the loop invariant becomes the full polynomial,
\[
y = \sum_{k=0}^{n}a_kx^k = a_0 + a_1x + a_2x^2 + \ldots + a_kx^k.
\]
Assuming the invariant holds true at the start of the iteration, the algorithm evaluates $y$ to the correct value.

\paragraph{d. Conclusion}
In conclusion, the loop invariant holds true at initialization, through maintainance all the way to termination by induction, and ends with a correct evaluation of the polynomial, so we can conclude that the algorithm is correct. 

\subsubsection*{Problem 2-4 Inversions}

\paragraph{a.}
The array $A=\langle 2, 3, 8, 6, 1 \rangle$ has the inversions $(4,5)$, $(3,5)$, $(2,5)$, $(1,5)$ and $(3,4)$.
\paragraph{b.}
The array with all the elements in reverse order, $A=\langle n, n-1, \ldots,2,1 \rangle$,  has the most \emph{inversions}. The last element $A[n]$ appears in $n-1$ inversions, since all other elements are greater and have a smaller index. Element $A[n-1]$ appears in $n-2$ inversions, since all elements to the left are greater and have a smaller index. So, starting from the left the number of inversions for each element in the array is, $n-1, n-2, \ldots, 1, 0$. So the total number of inversions in the inverted array are,
\[
\sum_{k=1}^{n} (k-1) = \sum_{k=1}^{n} k - n = \frac{n(n+1)}{2} - n = \frac{n(n - 1)}{2}, \quad \text{for} \quad n>0.
\]
\end{document}
\grid
